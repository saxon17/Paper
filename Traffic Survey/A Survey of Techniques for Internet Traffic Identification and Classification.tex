
%% bare_jrnl.tex
%% V1.3
%% 2007/01/11
%% by Michael Shell
%% see http://www.michaelshell.org/
%% for current contact information.
%%
%% This is a skeleton file demonstrating the use of IEEEtran.cls
%% (requires IEEEtran.cls version 1.7 or later) with an IEEE journal paper.
%%
%% Support sites:
%% http://www.michaelshell.org/tex/ieeetran/
%% http://www.ctan.org/tex-archive/macros/latex/contrib/IEEEtran/
%% and
%% http://www.ieee.org/



% *** Authors should verify (and, if needed, correct) their LaTeX system  ***
% *** with the testflow diagnostic prior to trusting their LaTeX platform ***
% *** with production work. IEEE's font choices can trigger bugs that do  ***
% *** not appear when using other class files.                            ***
% The testflow support page is at:
% http://www.michaelshell.org/tex/testflow/


%%*************************************************************************
%% Legal Notice:
%% This code is offered as-is without any warranty either expressed or
%% implied; without even the implied warranty of MERCHANTABILITY or
%% FITNESS FOR A PARTICULAR PURPOSE!
%% User assumes all risk.
%% In no event shall IEEE or any contributor to this code be liable for
%% any damages or losses, including, but not limited to, incidental,
%% consequential, or any other damages, resulting from the use or misuse
%% of any information contained here.
%%
%% All comments are the opinions of their respective authors and are not
%% necessarily endorsed by the IEEE.
%%
%% This work is distributed under the LaTeX Project Public License (LPPL)
%% ( http://www.latex-project.org/ ) version 1.3, and may be freely used,
%% distributed and modified. A copy of the LPPL, version 1.3, is included
%% in the base LaTeX documentation of all distributions of LaTeX released
%% 2003/12/01 or later.
%% Retain all contribution notices and credits.
%% ** Modified files should be clearly indicated as such, including  **
%% ** renaming them and changing author support contact information. **
%%
%% File list of work: IEEEtran.cls, IEEEtran_HOWTO.pdf, bare_adv.tex,
%%                    bare_conf.tex, bare_jrnl.tex, bare_jrnl_compsoc.tex
%%*************************************************************************

% Note that the a4paper option is mainly intended so that authors in
% countries using A4 can easily print to A4 and see how their papers will
% look in print - the typesetting of the document will not typically be
% affected with changes in paper size (but the bottom and side margins will).
% Use the testflow package mentioned above to verify correct handling of
% both paper sizes by the user's LaTeX system.
%
% Also note that the "draftcls" or "draftclsnofoot", not "draft", option
% should be used if it is desired that the figures are to be displayed in
% draft mode.
%
\documentclass[journal]{IEEEtran}
%
% If IEEEtran.cls has not been installed into the LaTeX system files,
% manually specify the path to it like:
% \documentclass[journal]{../sty/IEEEtran}





% Some very useful LaTeX packages include:
% (uncomment the ones you want to load)


% *** MISC UTILITY PACKAGES ***
%
%\usepackage{ifpdf}
% Heiko Oberdiek's ifpdf.sty is very useful if you need conditional
% compilation based on whether the output is pdf or dvi.
% usage:
% \ifpdf
%   % pdf code
% \else
%   % dvi code
% \fi
% The latest version of ifpdf.sty can be obtained from:
% http://www.ctan.org/tex-archive/macros/latex/contrib/oberdiek/
% Also, note that IEEEtran.cls V1.7 and later provides a builtin
% \ifCLASSINFOpdf conditional that works the same way.
% When switching from latex to pdflatex and vice-versa, the compiler may
% have to be run twice to clear warning/error messages.






% *** CITATION PACKAGES ***
%
%\usepackage{cite}
% cite.sty was written by Donald Arseneau
% V1.6 and later of IEEEtran pre-defines the format of the cite.sty package
% \cite{} output to follow that of IEEE. Loading the cite package will
% result in citation numbers being automatically sorted and properly
% "compressed/ranged". e.g., [1], [9], [2], [7], [5], [6] without using
% cite.sty will become [1], [2], [5]--[7], [9] using cite.sty. cite.sty's
% \cite will automatically add leading space, if needed. Use cite.sty's
% noadjust option (cite.sty V3.8 and later) if you want to turn this off.
% cite.sty is already installed on most LaTeX systems. Be sure and use
% version 4.0 (2003-05-27) and later if using hyperref.sty. cite.sty does
% not currently provide for hyperlinked citations.
% The latest version can be obtained at:
% http://www.ctan.org/tex-archive/macros/latex/contrib/cite/
% The documentation is contained in the cite.sty file itself.






% *** GRAPHICS RELATED PACKAGES ***
%
\ifCLASSINFOpdf
   \usepackage{graphicx}
  % declare the path(s) where your graphic files are
  % \graphicspath{{../pdf/}{../jpeg/}}
  % and their extensions so you won't have to specify these with
  % every instance of \includegraphics
  % \DeclareGraphicsExtensions{.pdf,.jpeg,.png}
\else
  % or other class option (dvipsone, dvipdf, if not using dvips). graphicx
  % will default to the driver specified in the system graphics.cfg if no
  % driver is specified.
  % \usepackage[dvips]{graphicx}
  % declare the path(s) where your graphic files are
  % \graphicspath{{../eps/}}
  % and their extensions so you won't have to specify these with
  % every instance of \includegraphics
  % \DeclareGraphicsExtensions{.eps}
\fi
% graphicx was written by David Carlisle and Sebastian Rahtz. It is
% required if you want graphics, photos, etc. graphicx.sty is already
% installed on most LaTeX systems. The latest version and documentation can
% be obtained at:
% http://www.ctan.org/tex-archive/macros/latex/required/graphics/
% Another good source of documentation is "Using Imported Graphics in
% LaTeX2e" by Keith Reckdahl which can be found as epslatex.ps or
% epslatex.pdf at: http://www.ctan.org/tex-archive/info/
%
% latex, and pdflatex in dvi mode, support graphics in encapsulated
% postscript (.eps) format. pdflatex in pdf mode supports graphics
% in .pdf, .jpeg, .png and .mps (metapost) formats. Users should ensure
% that all non-photo figures use a vector format (.eps, .pdf, .mps) and
% not a bitmapped formats (.jpeg, .png). IEEE frowns on bitmapped formats
% which can result in "jaggedy"/blurry rendering of lines and letters as
% well as large increases in file sizes.
%
% You can find documentation about the pdfTeX application at:
% http://www.tug.org/applications/pdftex





% *** MATH PACKAGES ***
%
%\usepackage[cmex10]{amsmath}
% A popular package from the American Mathematical Society that provides
% many useful and powerful commands for dealing with mathematics. If using
% it, be sure to load this package with the cmex10 option to ensure that
% only type 1 fonts will utilized at all point sizes. Without this option,
% it is possible that some math symbols, particularly those within
% footnotes, will be rendered in bitmap form which will result in a
% document that can not be IEEE Xplore compliant!
%
% Also, note that the amsmath package sets \interdisplaylinepenalty to 10000
% thus preventing page breaks from occurring within multiline equations. Use:
%\interdisplaylinepenalty=2500
% after loading amsmath to restore such page breaks as IEEEtran.cls normally
% does. amsmath.sty is already installed on most LaTeX systems. The latest
% version and documentation can be obtained at:
% http://www.ctan.org/tex-archive/macros/latex/required/amslatex/math/





% *** SPECIALIZED LIST PACKAGES ***
%
%\usepackage{algorithmic}
% algorithmic.sty was written by Peter Williams and Rogerio Brito.
% This package provides an algorithmic environment fo describing algorithms.
% You can use the algorithmic environment in-text or within a figure
% environment to provide for a floating algorithm. Do NOT use the algorithm
% floating environment provided by algorithm.sty (by the same authors) or
% algorithm2e.sty (by Christophe Fiorio) as IEEE does not use dedicated
% algorithm float types and packages that provide these will not provide
% correct IEEE style captions. The latest version and documentation of
% algorithmic.sty can be obtained at:
% http://www.ctan.org/tex-archive/macros/latex/contrib/algorithms/
% There is also a support site at:
% http://algorithms.berlios.de/index.html
% Also of interest may be the (relatively newer and more customizable)
% algorithmicx.sty package by Szasz Janos:
% http://www.ctan.org/tex-archive/macros/latex/contrib/algorithmicx/




% *** ALIGNMENT PACKAGES ***
%
%\usepackage{array}
% Frank Mittelbach's and David Carlisle's array.sty patches and improves
% the standard LaTeX2e array and tabular environments to provide better
% appearance and additional user controls. As the default LaTeX2e table
% generation code is lacking to the point of almost being broken with
% respect to the quality of the end results, all users are strongly
% advised to use an enhanced (at the very least that provided by array.sty)
% set of table tools. array.sty is already installed on most systems. The
% latest version and documentation can be obtained at:
% http://www.ctan.org/tex-archive/macros/latex/required/tools/


%\usepackage{mdwmath}
%\usepackage{mdwtab}
% Also highly recommended is Mark Wooding's extremely powerful MDW tools,
% especially mdwmath.sty and mdwtab.sty which are used to format equations
% and tables, respectively. The MDWtools set is already installed on most
% LaTeX systems. The lastest version and documentation is available at:
% http://www.ctan.org/tex-archive/macros/latex/contrib/mdwtools/


% IEEEtran contains the IEEEeqnarray family of commands that can be used to
% generate multiline equations as well as matrices, tables, etc., of high
% quality.


%\usepackage{eqparbox}
% Also of notable interest is Scott Pakin's eqparbox package for creating
% (automatically sized) equal width boxes - aka "natural width parboxes".
% Available at:
% http://www.ctan.org/tex-archive/macros/latex/contrib/eqparbox/





% *** SUBFIGURE PACKAGES ***
%\usepackage[tight,footnotesize]{subfigure}
% subfigure.sty was written by Steven Douglas Cochran. This package makes it
% easy to put subfigures in your figures. e.g., "Figure 1a and 1b". For IEEE
% work, it is a good idea to load it with the tight package option to reduce
% the amount of white space around the subfigures. subfigure.sty is already
% installed on most LaTeX systems. The latest version and documentation can
% be obtained at:
% http://www.ctan.org/tex-archive/obsolete/macros/latex/contrib/subfigure/
% subfigure.sty has been superceeded by subfig.sty.



%\usepackage[caption=false]{caption}
%\usepackage[font=footnotesize]{subfig}
% subfig.sty, also written by Steven Douglas Cochran, is the modern
% replacement for subfigure.sty. However, subfig.sty requires and
% automatically loads Axel Sommerfeldt's caption.sty which will override
% IEEEtran.cls handling of captions and this will result in nonIEEE style
% figure/table captions. To prevent this problem, be sure and preload
% caption.sty with its "caption=false" package option. This is will preserve
% IEEEtran.cls handing of captions. Version 1.3 (2005/06/28) and later
% (recommended due to many improvements over 1.2) of subfig.sty supports
% the caption=false option directly:
%\usepackage[caption=false,font=footnotesize]{subfig}
%
% The latest version and documentation can be obtained at:
% http://www.ctan.org/tex-archive/macros/latex/contrib/subfig/
% The latest version and documentation of caption.sty can be obtained at:
% http://www.ctan.org/tex-archive/macros/latex/contrib/caption/




% *** FLOAT PACKAGES ***
%
%\usepackage{fixltx2e}
% fixltx2e, the successor to the earlier fix2col.sty, was written by
% Frank Mittelbach and David Carlisle. This package corrects a few problems
% in the LaTeX2e kernel, the most notable of which is that in current
% LaTeX2e releases, the ordering of single and double column floats is not
% guaranteed to be preserved. Thus, an unpatched LaTeX2e can allow a
% single column figure to be placed prior to an earlier double column
% figure. The latest version and documentation can be found at:
% http://www.ctan.org/tex-archive/macros/latex/base/



%\usepackage{stfloats}
% stfloats.sty was written by Sigitas Tolusis. This package gives LaTeX2e
% the ability to do double column floats at the bottom of the page as well
% as the top. (e.g., "\begin{figure*}[!b]" is not normally possible in
% LaTeX2e). It also provides a command:
%\fnbelowfloat
% to enable the placement of footnotes below bottom floats (the standard
% LaTeX2e kernel puts them above bottom floats). This is an invasive package
% which rewrites many portions of the LaTeX2e float routines. It may not work
% with other packages that modify the LaTeX2e float routines. The latest
% version and documentation can be obtained at:
% http://www.ctan.org/tex-archive/macros/latex/contrib/sttools/
% Documentation is contained in the stfloats.sty comments as well as in the
% presfull.pdf file. Do not use the stfloats baselinefloat ability as IEEE
% does not allow \baselineskip to stretch. Authors submitting work to the
% IEEE should note that IEEE rarely uses double column equations and
% that authors should try to avoid such use. Do not be tempted to use the
% cuted.sty or midfloat.sty packages (also by Sigitas Tolusis) as IEEE does
% not format its papers in such ways.


%\ifCLASSOPTIONcaptionsoff
%  \usepackage[nomarkers]{endfloat}
% \let\MYoriglatexcaption\caption
% \renewcommand{\caption}[2][\relax]{\MYoriglatexcaption[#2]{#2}}
%\fi
% endfloat.sty was written by James Darrell McCauley and Jeff Goldberg.
% This package may be useful when used in conjunction with IEEEtran.cls'
% captionsoff option. Some IEEE journals/societies require that submissions
% have lists of figures/tables at the end of the paper and that
% figures/tables without any captions are placed on a page by themselves at
% the end of the document. If needed, the draftcls IEEEtran class option or
% \CLASSINPUTbaselinestretch interface can be used to increase the line
% spacing as well. Be sure and use the nomarkers option of endfloat to
% prevent endfloat from "marking" where the figures would have been placed
% in the text. The two hack lines of code above are a slight modification of
% that suggested by in the endfloat docs (section 8.3.1) to ensure that
% the full captions always appear in the list of figures/tables - even if
% the user used the short optional argument of \caption[]{}.
% IEEE papers do not typically make use of \caption[]'s optional argument,
% so this should not be an issue. A similar trick can be used to disable
% captions of packages such as subfig.sty that lack options to turn off
% the subcaptions:
% For subfig.sty:
% \let\MYorigsubfloat\subfloat
% \renewcommand{\subfloat}[2][\relax]{\MYorigsubfloat[]{#2}}
% For subfigure.sty:
% \let\MYorigsubfigure\subfigure
% \renewcommand{\subfigure}[2][\relax]{\MYorigsubfigure[]{#2}}
% However, the above trick will not work if both optional arguments of
% the \subfloat/subfig command are used. Furthermore, there needs to be a
% description of each subfigure *somewhere* and endfloat does not add
% subfigure captions to its list of figures. Thus, the best approach is to
% avoid the use of subfigure captions (many IEEE journals avoid them anyway)
% and instead reference/explain all the subfigures within the main caption.
% The latest version of endfloat.sty and its documentation can obtained at:
% http://www.ctan.org/tex-archive/macros/latex/contrib/endfloat/
%
% The IEEEtran \ifCLASSOPTIONcaptionsoff conditional can also be used
% later in the document, say, to conditionally put the References on a
% page by themselves.





% *** PDF, URL AND HYPERLINK PACKAGES ***
%
%\usepackage{url}
% url.sty was written by Donald Arseneau. It provides better support for
% handling and breaking URLs. url.sty is already installed on most LaTeX
% systems. The latest version can be obtained at:
% http://www.ctan.org/tex-archive/macros/latex/contrib/misc/
% Read the url.sty source comments for usage information. Basically,
% \url{my_url_here}.





% *** Do not adjust lengths that control margins, column widths, etc. ***
% *** Do not use packages that alter fonts (such as pslatex).         ***
% There should be no need to do such things with IEEEtran.cls V1.6 and later.
% (Unless specifically asked to do so by the journal or conference you plan
% to submit to, of course. )


% correct bad hyphenation here
\hyphenation{op-tical net-works semi-conduc-tor}


\begin{document}
%
% paper title
% can use linebreaks \\ within to get better formatting as desired
\title{A Survey of Techniques for Internet Traffic Identification and Classification}
%
%
% author names and IEEE memberships
% note positions of commas and nonbreaking spaces ( ~ ) LaTeX will not break
% a structure at a ~ so this keeps an author's name from being broken across
% two lines.
% use \thanks{} to gain access to the first footnote area
% a separate \thanks must be used for each paragraph as LaTeX2e's \thanks
% was not built to handle multiple paragraphs
%

\author{Mingwei~Wei,~\IEEEmembership{Member,~IEEE}% <-this % stops a space
\thanks{Mingwei Wei is with the Department
of Computer Science and Technology, Beijing Institute of Technology, Beijing,
100081 China (phone: 152-0161-3264; e-mail: weimw0417@163.com.)}
\thanks{Manuscript received July 25, 2015;}}

% note the % following the last \IEEEmembership and also \thanks -
% these prevent an unwanted space from occurring between the last author name
% and the end of the author line. i.e., if you had this:
%
% \author{....lastname \thanks{...} \thanks{...} }
%                     ^------------^------------^----Do not want these spaces!
%
% a space would be appended to the last name and could cause every name on that
% line to be shifted left slightly. This is one of those "LaTeX things". For
% instance, "\textbf{A} \textbf{B}" will typeset as "A B" not "AB". To get
% "AB" then you have to do: "\textbf{A}\textbf{B}"
% \thanks is no different in this regard, so shield the last } of each \thanks
% that ends a line with a % and do not let a space in before the next \thanks.
% Spaces after \IEEEmembership other than the last one are OK (and needed) as
% you are supposed to have spaces between the names. For what it is worth,
% this is a minor point as most people would not even notice if the said evil
% space somehow managed to creep in.



% The paper headers
\markboth{Journal of \LaTeX\ Class Files,~Vol.~6, No.~1, January~2007}%
{Shell \MakeLowercase{\textit{et al.}}: Bare Demo of IEEEtran.cls for Journals}
% The only time the second header will appear is for the odd numbered pages
% after the title page when using the twoside option.
%
% *** Note that you probably will NOT want to include the author's ***
% *** name in the headers of peer review papers.                   ***
% You can use \ifCLASSOPTIONpeerreview for conditional compilation here if
% you desire.




% If you want to put a publisher's ID mark on the page you can do it like
% this:
%\IEEEpubid{0000--0000/00\$00.00~\copyright~2007 IEEE}
% Remember, if you use this you must call \IEEEpubidadjcol in the second
% column for its text to clear the IEEEpubid mark.



% use for special paper notices
%\IEEEspecialpapernotice{(Invited Paper)}




% make the title area
\maketitle


\begin{abstract}
%\boldmath
The techniques for Internet traffic identification and classification are developed rapidly in recent years,
as it widely used in network management, monitor, design, security and research.
In the past decade, the traffic identification and classification techniques have been evolved along with development of Internet protocols and applications,
and many approaches have been proposed to optimize these techniques.
Nowadays, traffic measurement remains one of the hot areas in network research.
This is mostly based on the ever increasing network bandwidth, the growth number of network users, the constantly sophisticated applications and the development of technique about confusing traffic identification and classification.
In this paper, we present popular traffic identification and classification techniques, include port-based, payload-based, flow-based and host-based,
then analyze each technique from challenge aspect and make some remarks and recommendations that contribute to optimize traffic measurement.
\end{abstract}
% IEEEtran.cls defaults to using nonbold math in the Abstract.
% This preserves the distinction between vectors and scalars. However,
% if the journal you are submitting to favors bold math in the abstract,
% then you can use LaTeX's standard command \boldmath at the very start
% of the abstract to achieve this. Many IEEE journals frown on math
% in the abstract anyway.

% Note that keywords are not normally used for peerreview papers.
\begin{IEEEkeywords}
traffic identification, traffic classification, challenges, application detection, recommendations.
\end{IEEEkeywords}


% For peer review papers, you can put extra information on the cover
% page as needed:
% \ifCLASSOPTIONpeerreview
% \begin{center} \bfseries EDICS Category: 3-BBND \end{center}
% \fi
%
% For peerreview papers, this IEEEtran command inserts a page break and
% creates the second title. It will be ignored for other modes.
\IEEEpeerreviewmaketitle

\section{Introduction}
% The very first letter is a 2 line initial drop letter followed
% by the rest of the first word in caps.
%
% form to use if the first word consists of a single letter:
% \IEEEPARstart{A}{demo} file is ....
%
% form to use if you need the single drop letter followed by
% normal text (unknown if ever used by IEEE):
% \IEEEPARstart{A}{}demo file is ....
%
% Some journals put the first two words in caps:
% \IEEEPARstart{T}{his demo} file is ....
%
% Here we have the typical use of a "T" for an initial drop letter
% and "HIS" in caps to complete the first word.
\IEEEPARstart{W}{ith} the development of Internet technology and the advent of the era of mobile Internet, our life has been inseparable from the Internet nowadays.
According to the 35th statistical report on development of Internet in China published by CNNIC \cite{IEEEhowto:CNNIC}, the Internet users of China have reached 6.49 hundred million by the end of 2014.

\par
As shown in Fig. 1, the Internet population of China has been increasing rapidly in recent years, almost half of Chinese people are using Internet for work or daily life.

\begin{figure}[!ht]
\centering
\includegraphics[scale=0.41]{Figure/Internet_Popular_of_China.jpg}
\centering
\caption{Internet population of China}
\end{figure}

\par
As shown in Fig. 2, the Internet Penetration of China is higher and higher over the years, more and more people felt the charm of the Internet, and the Internet has penetrated into all walks of life of people.

\begin{figure}[!ht]
\centering
\includegraphics[scale=0.4]{Figure/Internet_Penetration_of_China.jpg}
\caption{Internet population of China}
\end{figure}

\par
Increasingly serious network security problems is in contrast with the rapid development of Internet technology.
In the past few years, constantly exposures of network security event make the network security problem get more and more attention.
If the security problem is not solved, especially in Internet Finance and Internet Payment, the further development and employment of Internet technology will be severely impeded.

\par
Traffic is the carrier of the Internet.
In large number of Internet traffic, a wide variety of malicious traffic is hidden.
These malicious traffic carrying viruses, trojans and worms threats the security of Internet.
It not only affects the network service provide's service quality, but also threatens the Internet user's privacy and data security, even the national security.
So how to find malicious traffic and intercept them is the challenge of Internet security.

\par
Traffic identification and classification is a technique that can detect applications the very traffic corresponded from mass of traffic.
Internet traffic identification and classification systems are deployed in gateway normally, it monitors traffic flows though gateway and intercept malicious traffic to ensure smooth operation of network.
Traffic identification and classification is basic of traffic control, this technique is widely used in network audit, content audit and intrusion detection, it plays a important role in increasing network management efficiency and guaranteeing network security.

\par
The traditional traffic identification techniques concentrate on content of traffic packet, so it is only to the recognition of unencrypted traffic effectively.
On this occasion, malicious users transfer illegal data with a safe data transmission protocol becomes possible.
Therefore, to identify legal and illegal data from encrypted traffic and classify traffic according to its type and source becomes a new challenge in network and Internet security.

\par
The techniques for Internet traffic identification and classification have been evolved along with development of Internet protocols and applications, and many techniques have been proposed, including port-based techniques \cite{mcpherson2004portvis}, payload-based techniques \cite{wang2004anomalous}, flow-based techniques \cite{kim2004flow}, host-based techniques \cite{karagiannis2005blinc} and graph-based techniques.
Some of them have been maturely and widely deployed in the current network, and some of them are still under researching.
But all the existing techniques still have their critical limitations and issues.
In addition, the existing techniques are facing more and more challenges as Internet protocols are becoming safer and applications are becoming more sophisticated.
In this paper, we first present popular traffic identification and classification techniques, then do in-depth analysis for them, especially their issues and challenges, and address some recommendations that can improve performance of techniques for Internet traffic identification and classification at last.

\par
The rest of the paper is organized as follows.
Section II explains port-based technique and analyzes its issues.
Section III explains payload-based technique and presents its drawbacks.
Section IV and Section V focus on flow-based and host-based techniques and dwell on their challenges.
Then it is followed by section VI with the general challenges of traffic identification and classification.
Section VII makes some final remarks on traffic analysis and provides some recommendations for solving the current issues.
Finally, we conclude the paper in section VIII.

\section{Analysis of Port-based Techniques}

The port-based technique is used to identify application according to TCP/UDP port number on transport layer protocol.
In the early stage of Internet, applications use specific port to set up communication.
Most of them used well-known port numbers assigned by IANA (The Internet Assigned Numbers Authority).
Table I shows the most popular well-known port number and its corresponding services and protocols.

\begin{table}[!ht]
\renewcommand\arraystretch{1.5}
\centering
\caption{popular well-known port numbers}
\begin{tabular}{c|c|c}
\hline \makebox[13em]{APPLICATION/SERVICE}&\makebox[7em]{PROTOCOL}&\makebox[7em]{PORT-NO}\\
\hline THUNDER&TCP/UDP&80,8000,8888\\
\hline QQ&UDP&4000\\
\hline SSH&TCP&22\\
\hline FTP&TCP&20,21\\
\hline WEB&TCP&80,443\\
\hline TELNET&TCP&23\\
\hline TOMCAT&TCP&8080\\
\hline
\end{tabular}
\end{table}

\par
To classify these applications or services, the port-based techniques only need to check source port number of IP data packet.
It finished tasks perfectly, as it assumes that most applications or services use well-known fixed TCP/UDP port number.
However, with the popularity of port jump technique, the port-based techniques lost their effectiveness.
The so called port jump technique refers to the applications use host port for communication without fixing port number or change port in the process of data transmission.
For example, utorrent can choose the random non-hold port for data transmission automatically when it has just started, this approach makes port-based technique failure.
Although port-based traffic identification and classification technique is easy to implement, it's also easy to circumvent.
So it is rarely to use as a main traffic identification technique now, often as an auxiliary means of other traffic identification and classification technology.

\section{Analysis of Payload-based Techniques}

In order to improve the low identification efficiency caused by the port jump technique, payload-based technique was proposed.
Deep Packet Inspection (DPI) is one of typical approach in payload-based techniques, it compares data content and characteristics of rule set which is built in advance with feature matching algorithm, and set results of matching as the basis of identification.
The item in characteristics of rule set includes type of data and feature string.
Table II shows the popular payload characteristics of applications.

\begin{table}[!ht]
\renewcommand\arraystretch{1.5}
\centering
\caption{payload characteristics}
\begin{tabular}{c|c}
\hline \makebox[9em]{PROTOCOL}&\makebox[19em]{PAYLOAD}\\
\hline HTTP&'GET' 'PUT' 'POST'\\
\hline SSH&'SSH'\\
\hline IRC&'USERHOST'\\
\hline Kazza&'Z-Kazza'\\
\hline BitTorrent&'\verb|\|x13BitTorrent protocol'\\
\hline QQ Voice&'SIP/user-agent: Tencent-VQ'\\
\hline Thunder&'\verb|\|x00\verb|\|x00\verb|\|x00\verb|\|x16\verb|\|x00\verb|\|x00\verb|\|x00\verb|\|x6a\verb|\|x01'\\
\hline eMule&'\verb|\|xe3,\verb|\|xc5,\verb|\|xd4,\verb|\|xe4,\verb|\|xe5,\verb|\|xf1'\\
\hline
\end{tabular}
\end{table}

\par
The precision and efficiency of DPI is decided by integrity of characteristics of rule set and feature matching algorithm, as shown in Fig. 3.
The feature matching algorithm includes one-mode and multi-mode.
One-mode refers to one scan can only match one feature string, such as Knuth-Morris-Pratt algorithm \cite{knuth1977fast} and Boyer-Moore algorithm \cite{boyer1977fast}.
Multi-mode refers to one scan can match a group of feature strings, such as Aho-Corasick algorithm \cite{aho1975efficient}, Commentz-Walter algorithm \cite{commentz1979string} and Aho-Corasick Boyer-Moore algorithm \cite{coit2001towards}.

\begin{figure}[!ht]
\centering
\includegraphics[scale=0.7]{Figure/Principle_of_Feature_Matching_Algorithm.jpg}
\caption{Principle of Feature Matching Algorithm}
\end{figure}

Although the payload-based technique can solve the issue of random port, they cannot deal with the encryption protocols, because the signatures of these types of protocols can hardly be found.
Generally, the main challenges resulting in issues are the encryption protocols, we list issues as follows.

\subsection{Fail in Encryption Protocol}

The encryption protocols get more and more attention in current Internet, as people have kept a watchful eye on the individual privacy.
More and more applications start to use the encryption protocols (e.g. SSL/TLS) for communication, such as Skype, Alipay, Evernote and Zhihu.
The signatures of packet payload can hardly be found as the specifications of these proprietary protocols remain private.
Therefore, the payload-based techniques lose their effectiveness when identifying and classifying such kinds of applications employing SSL/TLS protocols.

\subsection{Low Efficiency}

With the development of Internet bandwidth, the era of Terabit bandwidth has arrived.
Nowadays, the traffic identification and classification systems usually have to handle Gigabits or Terabits of data per second, which is a critical challenge task for payload-based techniques.
Due to payload-based techniques have to compare every packet content with characteristics of rule set until the protocol of the flow is determined, it needs higher processing capability to handle packets in high speed network.

Furthermore, as more and more new protocols appear, the size of characteristics of rule set becomes larger and larger, so the payload-based techniques have to compare more signatures or regular expressions, the processing efficiency of payload-based techniques will drop rapidly.

\section{analysis of flow-based techniques}

As the protocols and applications are continuing to evolve, port-based and payload-based techniques lose their effectiveness to the new coming protocols, the flow-based techniques are proposed under this circumstances.

\par
Just as its name implies, the flow-based technique is built on flows formed by a series of data packets.
It works by seeking and recording general law from data flow in a period of time.
For example, through the observation, we found most of P2P applications are using TCP protocol and UDP protocol at the same time.
Besides, it transmits control information by UDP and file content by TCP.
We use this law as the basis to identify the P2P application.
Different from the two techniques mentioned in section II and section III, the flow-based technique only analyzes the data flow and it's indifferent to the content of packet.
Therefore, the flow-based technique is suitable for common Internet traffic identification and classification as well as encrypted traffic.

\par
Most flow-based techniques use Machine Learning techniques as their identification or classification algorithms, such as Bayesian methods \cite{auld2007bayesian} and SVM Support Vector Machine \cite{yang2011smiler}.
Most flow-based techniques contain two main stages: training stage and classifying stage, as shown in Fig. 4 and Fig. 5 respectively.
At the training stage, the flow-based techniques use features extracted from the training data set to train identification and classification model.

\begin{figure}[!ht]
\centering
\includegraphics[scale=0.75]{Figure/Process_of_Training.jpg}
\caption{Process of Training Stage in Flow-based Techniques}
\end{figure}

\par
At the identification and classification stage, the flow-based techniques capture packets at first, then divide and refactor these packets to form flow.
After that, extracting features to matching models obtained from training stage to determine the type of protocols and applications.
In this way, it completes the traffic identification and classification task.

\begin{figure}[!ht]
\centering
\includegraphics[scale=0.66]{Figure/Process_of_Identification_or_Classification.jpg}
\caption{Process of Identification or Classification Stage in Flow-based Techniques}
\end{figure}

\par
In 2010 Yanfeng Sun et al. proposed a new identification method based on size of packets and clustering algorithm \cite{Yanfeng2010identification}.
They studied the methods of network application identification based on the transition pattern of payload length during the start up phase of the communication.
These methods didn't need to analyze all the packets, so they could identify the traffic at the early age of the communication.
But when analyzing more packets, they met ��the contents communication problem�� \cite{waizumi2007network}.
To solve this problem and adapt the change of the packets, they consulted ��the inverse method��, and used the inverse of the size of the packets.
And associating with the improved clustering algorithm K-medoids, they finally improved the validity of our identification method.

\section{analysis of host-based techniques}

\begin{figure}[!ht]
\centering
\includegraphics[scale=0.66]{Figure/Process_of_Training_host-based.jpg}
\caption{Process of Training Stage in Host-based Techniques}
\end{figure}

\begin{figure}[!ht]
\centering
\includegraphics[scale=0.66]{Figure/Process_of_Identification_or_Classification_host-based.jpg}
\caption{Process of Training Stage in Host-based Techniques}
\end{figure}

More specifically, in 2014 Maciej proposed stochastic fingerprints for application traffic flows conveyed in Secure Socket Layer/Transport Layer Security (SSL/TLS) sessions \cite{korczynski2014markov}.
The fingerprints are based on first-order homogeneous Markov chains for which it identifies the parameters from observed training application traces.
As the fingerprint parameters of chosen applications considerably differ, the method results in a very good accuracy of application discrimination and provides a possibility of detecting abnormal SSL/TLS sessions.

\par
We consider discrete-time random variable $X_t$ for any $t = t_0, t_1, ..., t_n \in T$.
It takes values $i_t \in \{1, ..., s\}$, where it is either an SSL/TLS message type (e.g. 22:2) or a sequence of the SSL/TLS message types transmitted in a single TCP segment (e.g. 22:11,22:14).

\par
We assume that $X_t$ is a first-order Markov chain \cite{campbell1998hidden}:

\begin{equation}
$$ $P(X_t = i_t|X_{t-1}) = i_{t-1}, X_{t-2} = i_{t-2}, ..., X_1 = i_1) = P(X_t = i_t|X_{t-1} = i_{t-1})$. $$
\end{equation}

We further assume that the Markov chain is homogeneous, i.e. a state transition from time $t-1$ to time $t$ is time-invariant:

\begin{equation}
$$ $P(X_t = i_t|X_{t-1} = i_{t_1}) = P(X_t = j|X_{t-1} = i) = p_{i-j}$, $$
\end{equation}

with the transition matrix \cite{campbell1998hidden}:

\begin{equation}
P = \left[
\begin{array}{lccr}
p_{1-1} & p_{1-2} & \cdots & p_{1-s}\\
p_{2-1} & p_{2-2} & \cdots & p_{2-s}\\
\vdots & \vdots & \ddots & \vdots\\
p_{s-1} & p_{s-2} & \cdots & p_{s-s}
\end{array}
\right],
\end{equation}

where: $\sum_{j=1}^{n} p_{i-j} = 1 $. We denote by:

\begin{equation}
Q = \left[ 
\begin{array}{lccr}
q_1,  &q_2,&\cdots,&q_s
\end{array}
\right],
\end{equation}

the ENter Probability Distribution (ENPD), where $q_i = P(X_t = i)$ at time $t_0$, and we define:

\begin{equation}
W = \left[
\begin{array}{lccr}
w_1,  &w_2,&\cdots,&w_s
\end{array}
\right],
\end{equation}

\subsection{Subsection Heading Here}
Subsection text here.

% needed in second column of first page if using \IEEEpubid
%\IEEEpubidadjcol

\subsubsection{Subsubsection Heading Here}
Subsubsection text here.


% An example of a floating figure using the graphicx package.
% Note that \label must occur AFTER (or within) \caption.
% For figures, \caption should occur after the \includegraphics.
% Note that IEEEtran v1.7 and later has special internal code that
% is designed to preserve the operation of \label within \caption
% even when the captionsoff option is in effect. However, because
% of issues like this, it may be the safest practice to put all your
% \label just after \caption rather than within \caption{}.
%
% Reminder: the "draftcls" or "draftclsnofoot", not "draft", class
% option should be used if it is desired that the figures are to be
% displayed while in draft mode.
%
%\begin{figure}[!t]
%\centering
%\includegraphics[width=2.5in]{myfigure}
% where an .eps filename suffix will be assumed under latex,
% and a .pdf suffix will be assumed for pdflatex; or what has been declared
% via \DeclareGraphicsExtensions.
%\caption{Simulation Results}
%\label{fig_sim}
%\end{figure}

% Note that IEEE typically puts floats only at the top, even when this
% results in a large percentage of a column being occupied by floats.


% An example of a double column floating figure using two subfigures.
% (The subfig.sty package must be loaded for this to work.)
% The subfigure \label commands are set within each subfloat command, the
% \label for the overall figure must come after \caption.
% \hfil must be used as a separator to get equal spacing.
% The subfigure.sty package works much the same way, except \subfigure is
% used instead of \subfloat.
%
%\begin{figure*}[!t]
%\centerline{\subfloat[Case I]\includegraphics[width=2.5in]{subfigcase1}%
%\label{fig_first_case}}
%\hfil
%\subfloat[Case II]{\includegraphics[width=2.5in]{subfigcase2}%
%\label{fig_second_case}}}
%\caption{Simulation results}
%\label{fig_sim}
%\end{figure*}
%
% Note that often IEEE papers with subfigures do not employ subfigure
% captions (using the optional argument to \subfloat), but instead will
% reference/describe all of them (a), (b), etc., within the main caption.


% An example of a floating table. Note that, for IEEE style tables, the
% \caption command should come BEFORE the table. Table text will default to
% \footnotesize as IEEE normally uses this smaller font for tables.
% The \label must come after \caption as always.
%
%\begin{table}[!t]
%% increase table row spacing, adjust to taste
%\renewcommand{\arraystretch}{1.3}
% if using array.sty, it might be a good idea to tweak the value of
% \extrarowheight as needed to properly center the text within the cells
%\caption{An Example of a Table}
%\label{table_example}
%\centering
%% Some packages, such as MDW tools, offer better commands for making tables
%% than the plain LaTeX2e tabular which is used here.
%\begin{tabular}{|c||c|}
%\hline
%One & Two\\
%\hline
%Three & Four\\
%\hline
%\end{tabular}
%\end{table}


% Note that IEEE does not put floats in the very first column - or typically
% anywhere on the first page for that matter. Also, in-text middle ("here")
% positioning is not used. Most IEEE journals use top floats exclusively.
% Note that, LaTeX2e, unlike IEEE journals, places footnotes above bottom
% floats. This can be corrected via the \fnbelowfloat command of the
% stfloats package.



\section{Conclusion}
The conclusion goes here.





% if have a single appendix:
%\appendix[Proof of the Zonklar Equations]
% or
%\appendix  % for no appendix heading
% do not use \section anymore after \appendix, only \section*
% is possibly needed

% use appendices with more than one appendix
% then use \section to start each appendix
% you must declare a \section before using any
% \subsection or using \label (\appendices by itself
% starts a section numbered zero.)
%


\appendices
\section{Proof of the First Zonklar Equation}
Appendix one text goes here.

% you can choose not to have a title for an appendix
% if you want by leaving the argument blank
\section{}
Appendix two text goes here.


% use section* for acknowledgement
\section*{Acknowledgment}


The authors would like to thank...


% Can use something like this to put references on a page
% by themselves when using endfloat and the captionsoff option.
\ifCLASSOPTIONcaptionsoff
  \newpage
\fi



% trigger a \newpage just before the given reference
% number - used to balance the columns on the last page
% adjust value as needed - may need to be readjusted if
% the document is modified later
%\IEEEtriggeratref{8}
% The "triggered" command can be changed if desired:
%\IEEEtriggercmd{\enlargethispage{-5in}}

% references section

% can use a bibliography generated by BibTeX as a .bbl file
% BibTeX documentation can be easily obtained at:
% http://www.ctan.org/tex-archive/biblio/bibtex/contrib/doc/
% The IEEEtran BibTeX style support page is at:
% http://www.michaelshell.org/tex/ieeetran/bibtex/
%\bibliographystyle{IEEEtran}
% argument is your BibTeX string definitions and bibliography database(s)
%\bibliography{IEEEabrv,../bib/paper}
%
% <OR> manually copy in the resultant .bbl file
% set second argument of \begin to the number of references
% (used to reserve space for the reference number labels box)
\begin{thebibliography}{20}

\bibitem{IEEEhowto:CNNIC}
CNNIC: Statistics Report of Development of China Internet Network. (Jan 2015). [Online]. Available:
http://www.cnnic.cn/hlwfzyj/hlwxzbg /201502/P020150203551802054676.pdf

\bibitem{mcpherson2004portvis}
J.~McPherson, K.-L. Ma, P.~Krystosk, T.~Bartoletti, and M.~Christensen,
  ``Portvis: a tool for port-based detection of security events,'' in {\em
  Proceedings of the 2004 ACM workshop on Visualization and data mining for
  computer security}, pp.~73--81, ACM, 2004.

\bibitem{wang2004anomalous}
K.~Wang and S.~J. Stolfo, ``Anomalous payload-based network intrusion
  detection,'' in {\em Recent Advances in Intrusion Detection}, pp.~203--222,
  Springer, 2004.

\bibitem{kim2004flow}
A.-S. Kim, H.-J. Kong, S.-C. Hong, S.-H. Chung, and J.~W. Hong, ``A flow-based
  method for abnormal network traffic detection,'' in {\em Network operations
  and management symposium, 2004. NOMS 2004. IEEE/IFIP}, vol.~1, pp.~599--612,
  IEEE, 2004.

\bibitem{karagiannis2005blinc}
T.~Karagiannis, K.~Papagiannaki, and M.~Faloutsos, ``Blinc: multilevel traffic
  classification in the dark,'' in {\em ACM SIGCOMM Computer Communication
  Review}, vol.~35, pp.~229--240, ACM, 2005.

\bibitem{knuth1977fast}
D.~E. Knuth, J.~H. Morris, Jr, and V.~R. Pratt, ``Fast pattern matching in
  strings,'' {\em SIAM journal on computing}, vol.~6, no.~2, pp.~323--350,
  1977.

\bibitem{boyer1977fast}
R.~S. Boyer and J.~S. Moore, ``A fast string searching algorithm,'' {\em
  Communications of the ACM}, vol.~20, no.~10, pp.~762--772, 1977.

\bibitem{aho1975efficient}
A.~V. Aho and M.~J. Corasick, ``Efficient string matching: an aid to
  bibliographic search,'' {\em Communications of the ACM}, vol.~18, no.~6,
  pp.~333--340, 1975.

\bibitem{commentz1979string}
B.~Commentz-Walter, {\em A string matching algorithm fast on the average}.
\newblock Springer, 1979.

\bibitem{coit2001towards}
C.~J. Coit, S.~Staniford, and J.~McAlerney, ``Towards faster string matching
  for intrusion detection or exceeding the speed of snort,'' in {\em DARPA
  Information Survivability Conference \&amp; Exposition II, 2001. DISCEX'01.
  Proceedings}, vol.~1, pp.~367--373, IEEE, 2001.

\bibitem{auld2007bayesian}
T.~Auld, A.~W. Moore, and S.~F. Gull, ``Bayesian neural networks for internet
  traffic classification,'' {\em Neural Networks, IEEE Transactions on},
  vol.~18, no.~1, pp.~223--239, 2007.

\bibitem{yang2011smiler}
B.~Yang, G.~Hou, L.~Ruan, Y.~Xue, and J.~Li, ``Smiler: towards practical online
  traffic classification,'' in {\em Proceedings of the 2011 ACM/IEEE Seventh
  Symposium on Architectures for Networking and Communications Systems},
  pp.~178--188, IEEE Computer Society, 2011.

\bibitem{korczynski2014markov}
M.~Korczynski and A.~Duda, ``Markov chain fingerprinting to classify encrypted
  traffic,'' in {\em INFOCOM, 2014 Proceedings IEEE}, pp.~781--789, IEEE, 2014.

\bibitem{campbell1998hidden}
M.~Campbell, ``Hidden markov and other models for discrete-valued time
  series,'' {\em Biometrics}, vol.~54, no.~1, p.~394, 1998.

\bibitem{Yanfeng2010identification}
Y.~Sun and S.~Zhang, ``a new identification method based on size of packets and
  clustering algorithm,'' {\em Telecommunications Information}, no.~2,
  pp.~26--28, 2010.

\bibitem{waizumi2007network}
Y.~Waizumi, A.~Jamalipour, and Y.~Nemoto, ``Network application identification
  based on transition pattern of packets,'' in {\em IEEE Wireless Rural and
  Emergency Communications Conference (WRECOM) 2007}, 2007.
  
\end{thebibliography}

% biography section
%
% If you have an EPS/PDF photo (graphicx package needed) extra braces are
% needed around the contents of the optional argument to biography to prevent
% the LaTeX parser from getting confused when it sees the complicated
% \includegraphics command within an optional argument. (You could create
% your own custom macro containing the \includegraphics command to make things
% simpler here.)
%\begin{biography}[{\includegraphics[width=1in,height=1.25in,clip,keepaspectratio]{mshell}}]{Michael Shell}
% or if you just want to reserve a space for a photo:

\begin{IEEEbiography}[{\includegraphics[width=1in,height=1.25in,clip,keepaspectratio]{author}}]{Mingwei Wei,}
is the corresponding author of this paper.
He is working on master's degree in Network and Information Security Lab in school of computing in Beijing Institute of Technology.
His tutor are Mingzhong Wang and Liehuang Zhu.
His research interests include network security, business process management and traffic analysis.
\end{IEEEbiography}

% if you will not have a photo at all:

% insert where needed to balance the two columns on the last page with
% biographies
%\newpage

% You can push biographies down or up by placing
% a \vfill before or after them. The appropriate
% use of \vfill depends on what kind of text is
% on the last page and whether or not the columns
% are being equalized.

%\vfill

% Can be used to pull up biographies so that the bottom of the last one
% is flush with the other column.
%\enlargethispage{-5in}



% that's all folks
\end{document}


